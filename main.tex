\documentclass[a4paper]{article}
\usepackage[utf8]{inputenc}
\usepackage[ngerman]{babel}
\usepackage[babel,german=quotes]{csquotes} 

\title{Open Glider Network}
\author{Konstantin Gründger}
\date{}

\begin{document}

\maketitle

\tableofcontents


\section{Einleitung}

\section{Die OGN-Infrastruktur}
\subsection{Sender}
\subsection{Empfänger}
\subsection{Server}
\subsection{Anwendungen}

\section{Aufbau des Empfängers}
\subsection{Standortwahl}
\subsection{Antenne}
\subsection{Verstärker}
\subsection{Filter}

\section{Inbetriebnahme}
\subsection{Softwareinstallation}
\subsection{Einstellung}



\section{Fehlersuche}
\subsection{Fernwartung}
Um den Aufwand bei der Fehleranalyse und -behebung gering zu halten, sollte man die Möglichkeit haben, aus dem Internet auf den Raspberry zuzugreifen. Dieser Zugriff sollte möglichst stabil und sicher sein.
\begin{description}
\item[Port forwarding] Wenn der Router, an dem der Raspberry hängt, von aussen erreichbar ist, dann ist der einfachste Weg, wenn man die benötigten Ports auf den Raspberry weiterleitet.
\item[VPN] Das ist am besten 
\end{description}
\subsection{Frequenzanalyse}

\section{Rechtliches}

\end{document}
